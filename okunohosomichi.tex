\section{奥の細道}
01 \ruby[g]{序文}{じょぶん}

\ruby[g]{月日}{つきひ}は\ruby[g]{百代}{はくたい}の\ruby[g]{過客}{かかく}にして、\ruby[g]{行}{ゆ}きかふ年もまた\ruby[g]{旅人}{たびびと}なり。

舟の上に\ruby[g]{生涯}{しょうがい}をうかべ、馬の口とらえて\ruby[g]{老}{おい}をむかふるものは、\ruby[g]{日々}{ひび}\ruby[g]{旅}{たび}にして\ruby[g]{旅}{たび}を\ruby[g]{栖}{すみか}とす。

\ruby[g]{古人}{こじん}も多く\ruby[g]{旅}{たび}に\ruby[g]{死}{し}せるあり。

よもいづれの年よりか、\ruby[g]{片雲}{へんうん}の風にさそはれて、\ruby[g]{漂泊}{ひょうはく}の思ひやまず、\ruby[g]{海浜}{かいひん}にさすらへ、\ruby[g]{去年}{こぞ}の\ruby[g]{秋江上}{こうしょう}の\ruby[g]{破屋}{はおく}にくもの\ruby[g]{古巣}{ふるす}をはらひて、やや年も\ruby[g]{暮}{くれ}、春立てる\ruby[g]{霞}{かすみ}の空に\ruby[g]{白河}{しらかわ}の関こえんと、そぞろ\ruby[g]{神}{がみ}の物につきて心をくるはせ、\ruby[g]{道祖神}{どうそじん}のまねきにあひて、\ruby[g]{取}{と}るもの手につかず。

ももひきの\ruby[g]{破}{やぶ}れをつづり、\ruby[g]{笠}{かさ}の\ruby[g]{緒}{お}\ruby[g]{付}{つ}けかえて、\ruby[g]{三里}{さんり}に\ruby[g]{灸}{きゅう}すゆるより、松島の月まず心にかかりて、\ruby[g]{住}{す}める\ruby[g]{方}{かた}は人に\ruby[g]{譲}{ゆず}り、\ruby[g]{杉風}{さんぷう}が\ruby[g]{別墅}{べっしょ}に\ruby[g]{移}{うつ}るに、

  草の戸も \ruby[g]{住替}{すみかわる}る\ruby[g]{代}{よ}ぞ ひなの家

\ruby[g]{面八句}{おもてはちく}を\ruby[g]{庵}{いおり}の\ruby[g]{柱}{はしら}にかけ\ruby[g]{置}{お}く。

02 \ruby[g]{旅立}{たびだ}ち

\ruby[g]{弥生}{やよい}も\ruby[g]{末}{すえ}の七日、あけぼのの空\ruby[g]{朧々}{ろうろう}として、月はありあけにて光おさまれるものから、\ruby[g]{富士}{ふじ}の\ruby[g]{嶺}{みね}かすかに見えて、\ruby[g]{上野}{うえの}・\ruby[g]{谷中}{やなか}の花の\ruby[g]{梢}{こずえ}、またいつかはと心ぼそし。

むつましきかぎりは\ruby[g]{宵}{よい}よりつどひて、舟に\ruby[g]{乗}{の}りて送る。

千じゆといふ所にて舟をあがれば、\ruby[g]{前途}{せんど}\ruby[g]{三千里}{さんぜんり}の思い\ruby[g]{胸}{むね}にふさがりて、\ruby[g]{幻}{まぼろし}のちまたに\ruby[g]{離別}{りべつ}の\ruby[g]{泪}{なみだ}をそそぐ。

  \ruby[g]{行}{ゆ}く春や \ruby[g]{鳥啼}{とりなき}\ruby[g]{魚}{うお}の 目は\ruby[g]{泪}{なみだ}

これを\ruby[g]{矢立}{やたて}の\ruby[g]{初}{はじめ}として、\ruby[g]{行}{ゆ}く道なを進まず。

人々は\ruby[g]{途中}{みちなか}に\ruby[g]{立}{た}ちならびて、\ruby[g]{後}{うし}ろかげの見ゆるまではと\ruby[g]{見送}{みおく}るなるべし。