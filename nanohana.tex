\section{\ruby[g]{菜}{な}の花と小娘 作:\ruby[g]{志賀}{しが}\ruby[g]{直哉}{なおや}}

\ruby[g]{或}{あ}る晴れた静かな春の日の午後でした。


一人の\ruby[g]{小}{こ}\ruby[g]{娘}{むすめ}が山で枯れ枝を\ruby[g]{拾}{ひろ}っていました。

やがて、夕日が\ruby[g]{新}{しん}\ruby[g]{緑}{りょく}の薄い\ruby[g]{木}{こ}の葉を\ruby[g]{透}{す}かして赤々と見られる頃、小娘は集めた小枝を、小さい\ruby[g]{草}{くさ}\ruby[g]{原}{はら}に持ち出して、そこで自分の\ruby[g]{背}{せ}\ruby[g]{負}{お}ってきた荒い\ruby[g]{目}{め}\ruby[g]{籠}{かご}に\ruby[g]{詰}{つ}めはじめました。

そうして、しばらくたちました。

すると、小娘はふと誰かに自分が\ruby[g]{呼}{よ}ばれたような気がしました。

「ええ?」小娘は思わずそう言って、立ってそのへんを見回しました。

が、そこには誰の姿も見えません。

「誰?私を呼ぶの。」小娘はもう一度大きい声でこう言ってみましたが、\ruby[g]{矢}{や}\ruby[g]{張}{は}り答える\ruby[g]{者}{もの}はありませんでした。

\ruby[g]{二三度}{にさんど}そんな気がして、初めて気がつくと、それは雑草の中からただ\ruby[g]{一本}{ひともと}、わずかに首を出していた\ruby[g]{憐}{あわ}れな小さい\ruby[g]{菜}{な}の花でした。

小娘は頭にかぶっていた手ぬぐいを取って、顔の汗を\ruby[g]{拭}{ふ}き拭き近寄って行きました。



20131107001.jpg



「お前、こんなところで、よくさびしくないのね。」

「さびしいわ。」と菜の花は親しげに答えました。

「そんならならなぜ来たのさ。」小娘は\ruby[g]{叱}{しか}りでもするような調子で言いました。

すると菜の花は、「ひばりの\ruby[g]{胸}{むな}\ruby[g]{毛}{げ}に着いてきた種が、ここでこぼれたのよ。困るわ。」と悲しげに答えました。

そして、どうか私をお仲間の多い\ruby[g]{麓}{ふもと}の村へ連れていってくださいと頼みました。

小娘は可哀そうに思いました。

小娘は菜の花の願いを、かなえてやろうと考えました。

そして\ruby[g]{静}{しず}かにそれを根から抜くと、自分の\ruby[g]{荷物}{にもつ}を\ruby[g]{背負}{せお}い、それを片手に持って、\ruby[g]{山路}{やまじ}を村の方へと\ruby[g]{下}{くだ}って行きました。



清い小さな流れが、水音をたてて、その路にそうて流れていました。

「あなたの手は随分、ほてるのね。」

しばらくすると、手の菜の花は\ruby[g]{不}{ふ}\ruby[g]{意}{い}にこんなことを言い出しました。

「あつい手で持たれると、首がだるくなって仕方がないわ、まっすぐにしていられなくなるわ。」

そう言いながら、菜の花はうなだれた首を小娘の\ruby[g]{歩調}{ほ ちょう}に合せ、力なく振っていました。

小娘は、ちょっと\ruby[g]{当惑}{とうわく}しました。

そして、心配そうに、「苦しいの?」と下を向いてしまった菜の花を、のぞき込んで言いました。

「そんなでもないの、いいの。心配なさらないでも。」

菜の花は苦しいのを\ruby[g]{我}{が}\ruby[g]{慢}{まん}して答えました。

小娘には\ruby[g]{図}{はか}らず、いい考えが浮かびました。

「いい!いい!」と小娘は言いました。

そうして\ruby[g]{身軽}{みがる}く\ruby[g]{道端}{みちばた}にしゃがむと、そのまま\ruby[g]{黙}{だま}って菜の花の根を流れへ\ruby[g]{浸}{ひた}してやりました。

「まあ!」

菜の花は生き返ったような元気な声を出して小娘を見上げました。

すると、小娘は\ruby[g]{宣告}{せんこく}するように、「このまま流れて行くのよ。」と言いました。

菜の花は\ruby[g]{不安}{ふあん}そうに首を振りました。

「先に流れてしまうと恐いわ。」

「大丈夫。心配しなくてもいいの。」

そう言いながら、早くも小娘は流れの\ruby[g]{表面}{ひょうめん}で、持っていた菜の花を離してしまいました。

「恐いは、恐いわ。」と流れの水にさらわれながら、菜の花は見る見る小娘から遠くなるのを\ruby[g]{心配}{しんぱい}そうに叫びました。

が、小娘は\ruby[g]{黙}{だま}って立ち上がると、両手を後へ\ruby[g]{回}{まわ}し、\ruby[g]{背}{せ}で\ruby[g]{跳}{おど}る目籠をおさえ、駆けて来ました。



菜の花は安心しました。そして、さも\ruby[g]{嬉}{うれ}しそうに水面から小娘を見上げて、何かと話しかけるのでした。

どこからともなく\ruby[g]{気軽}{きがる}な\ruby[g]{黄蝶}{きちょう}が飛んできました。

そして、うるさく菜の花の上をついて飛んできました。

菜の花はそれを大\ruby[g]{変嬉}{うれ}しがっていました。

しかし黄蝶は、せっかちで、移り気でした。

そして、いつとはなしに、又、どこかへ飛んでいってしまいました。

菜の花は小娘の\ruby[g]{鼻}{はな}の頭にポツポツと玉のような汗が浮かび出しているのに気がつきました。

「今度はあなたが苦しいわ。」と菜の花は心配そうに言いました。

が、小娘はかえって「心配しなくてもいいのよ。」と不\ruby[g]{愛想}{ぶあいそう}に答えました。

菜の花は、\ruby[g]{叱}{しか}られたのかと思って、黙ってしまいました。



間もなく小娘は菜の花の\ruby[g]{悲鳴}{ひめい}に\ruby[g]{驚}{おどろ}かされました。

菜の花は流れに波打っている髪の毛のような水草に、根をからまれて、さも苦しげに首をふっていました。

「まあ、少しそうしてお休み。」

小娘は息をはずませながら、\ruby[g]{傍}{かたわ}らの石に腰をおろしました。

「こんなものに足をからまれて休むの、\ruby[g]{気}{き}\ruby[g]{持}{もち}が悪いわ。」

そう言いながら、菜の花は\ruby[g]{尚}{なお}しきりにいやいやをしておりました。

「それで、いいのよ。」小娘は汗ばんだ真っ赤な顔に意\ruby[g]{地悪}{いじわる}な、しかし\ruby[g]{親}{した}しみのある笑いを浮かべて言いました。

「いやなの。休むのはいいけど、こうしているのは気持が悪いの、どうか\ruby[g]{一寸}{ちょっと}あげてちょうだい。どうか。」

「いいのよ。」小娘は笑って取り合いません。

が、そのうち水のいきおいで菜の花の根は自然に水草から、すり抜けて行きました。

そして、不意に、「流れるぅ!」と、大きな声を出して菜の花はまた、流されて行きました。

小娘も急いで立ち上がると、それを追って駆け出しました。

少しきたところで、「やっぱりあなたが苦しいわ。」と菜の花はこわごわ言いました。

「何でもないの、心配しなくてもいいの。」

今度は小娘も優しく答えてやりました。

そうして、菜の花に気をもませまいと、わざと菜の花より二\ruby[g]{三間}{にさんげん}先を\ruby[g]{駆}{か}けて行くことにしました。



20131107002.jpg



\ruby[g]{麓}{ふもと}の村が見えてきました。

小娘はふり\ruby[g]{返}{かえ}らずに、「もうすぐよ。」と声をかけました。

「そう。」と、後で菜の花が言いました。

それきりしばらく話は\ruby[g]{絶}{た}えました。

ただ流れの水音にまじって、バタバタ、バタバタ、という小娘の\ruby[g]{草履}{ぞうり}で走る足音が聞こえていました。

ポチャーンという\ruby[g]{水音}{みずおと}がしました。

と、すぐ、小娘は菜の花の死にそうな\ruby[g]{悲鳴}{ひめい}を\ruby[g]{聴}{き}きました。

小娘は驚いて立ち止まりました。

見ると菜の花は、花も葉も色がさめたようになって、「早く早く。」と\ruby[g]{延}{の}びあがっています。

小娘は急いで引き上げてやりました。

「どうしたのよ。」

小娘はその胸に菜の花を抱くようにして、後の流れを見回しながら\ruby[g]{訊}{き}きました。

「あなたの\ruby[g]{足元}{あしもと}から何か\ruby[g]{飛}{と}び込んだのよ。」

菜の花はまだ\ruby[g]{動悸}{どうき}が\ruby[g]{治}{おさ}まらないように、言葉を\ruby[g]{切}{き}りました。

「いぼ\ruby[g]{蛙}{かえる}なのよ。一度もぐって、不意に私の顔の前に、浮かび上がったのよ。口の\ruby[g]{尖}{とが}った\ruby[g]{意地}{いじ}の悪そうな、あの\ruby[g]{河童}{かっぱ}のような顔に、もう少しで、頬っぺたをドスンとぶつけるところでしたわ。」

それを聴いて小娘は、大きな声をして笑いました。

「笑い\ruby[g]{事}{ごと}じゃあ、ないわ。」と菜の花はうらめしそうに言いました。

「でも、私が思わず大きな声をしたら、今度は蛙の方でびっくりして、あわててもぐってしまいましたわ。」

こう言って菜の花も笑いました。



間もなく村へ着きました。

小娘は\ruby[g]{早速}{さっそく}自分の家の\ruby[g]{花畑}{はなばたけ}に\ruby[g]{一緒}{いっしょ}にそれを植えてやりました。

そこは山の\ruby[g]{雑草}{ざっそう}の中とはちがって土がよく\ruby[g]{肥}{こ}えておりました。

菜の花はどんどん延び育ちました。

そうして、今は\ruby[g]{多勢}{おおぜい}の\ruby[g]{仲間}{なかま}と仲よく、\ruby[g]{仕合}{しあわ}せに\ruby[g]{暮}{く}らせる身となりました。

