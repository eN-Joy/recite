\section{过秦论}


  秦孝公据殽函之固,拥雍州之地,君臣固守,以窥周室;有席卷天下,包举宇内,囊括四海之意,并吞八荒之心。当是时也,商君佐之;内立法度,务耕织,修守战之具;外连衡而鬬诸侯。于是秦人拱手而取西河之外。孝公既没,惠文、武、昭,蒙故业,因遗策,南取汉中,西举巴蜀,东割膏腴之地,北收要害之郡。诸侯恐惧,会盟而谋弱秦,不爱珍器重宝、肥饶之地,以致天下之士,合从缔交,相与为一。当此之时,齐有孟尝,赵有平原,楚有春申,魏有信陵;此四君者,皆明智而忠信,宽厚而爱人,尊贤重士,约从离横,兼韩、魏、燕、赵、宋、卫、中山之众,于是六国之士,有甯越、徐尚、苏秦、杜赫之属为之谋,齐明、周最、陈轸、昭滑、楼绥、翟景、苏厉、乐毅之徒通其意,吴起、孙膑、带佗、儿良、王廖、田忌、廉颇、赵奢之伦制其兵;尝以什倍之地,百万之众,叩关而攻秦。秦人开关延敌,九国之师逡巡遁逃而不敢进。秦无亡矢遗镞之费,而天下诸侯已困矣。于是从散约解,争割地而赂秦。秦有馀力而制其敝,追亡逐北,伏尸百万,流血漂橹;因利乘便,宰割天下,分裂河山,强国请服,弱国入朝。施及孝文王、庄襄王,享国之日浅,国家无事。


  及至始皇,奋六世之馀烈,振长策而御宇内,吞二周而亡诸侯,履至尊而制六合,执棰拊以鞭笞天下,威振四海。南取百越之地,以为桂林、象郡;百越之君,俛首系颈,委命下吏。乃使蒙恬北筑长城,而守藩篱,却匈奴七百馀里;胡人不敢南下而牧马,士不敢弯弓而报怨。于是废先王之道,燔百家之言,以愚黔首;隳名城,杀豪俊,收天下之兵,聚之咸阳,销锋铸鐻,以为金人十二,以弱天下之民。然后践华为城,因河为池,据亿丈之城,临不测之渊以为固。良将劲弩,守要害之处;信臣精卒,陈利兵而谁何!天下已定,始皇之心,自以为关中之固,金城千里,子孙帝王万世之业也。


  始皇既没,馀威震于殊俗。然陈涉,瓮牖绳枢之子,氓隶之人,而迁徙之徒也。材能不及中人,非有仲尼、墨翟之贤,陶朱、猗顿之富,蹑足行伍之闲,崛起阡陌之中,率罢弊之卒,将数百之众,转而攻秦。斩木为兵,揭竿为旗,天下云集而响应,赢粮而景从,山东豪俊,遂并起而亡秦族矣。


  且夫天下非小弱也,雍州之地,殽函之固,自若也。陈涉之位,非尊于齐、楚、燕、赵、韩、魏、宋、卫、中山之君也;锄耰棘矜,非铦于钩戟长铩也;谪戍之众,非抗于九国之师也;深谋远虑,行军用兵之道,非及曩时之士也。然而成败异变,功业相反。试使山东之国,与陈涉度长絜大,比权量力,则不可同年而语矣。然秦以区区之地,致万乘之权,招八州而朝同列,百有馀年矣。然后以六合为家,殽函为宫,一夫作难,而七庙隳,身死人手,为天下笑者,何也?仁义不施,而攻守之势异也。


