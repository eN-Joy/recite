\section{石臼の歌}
田舎では自分の家で石臼を回して小麦やお米を粉にし、団子やうどんを作ります。千枝子はごろごろっという石臼の歌が大好き。八月になると、千枝子のお父さんの弟の子供である瑞枝が、広島からやってきました。ところが、八月六日、広島には原爆が落とされ、広島に残っていた瑞枝の両親は一度に亡くなってしまいます。

もう明日はお盆の十三日。お墓の掃除をして、魂をお迎えせねばなりません。

しかし、お婆さんの碾き臼は、一向に動きませんでした。薄の前に座ったまま、言葉少なく考え込んでいるお婆さんのそばで、臼は黙ってないているのでしょうか。

「おばあさん、私は引くわ。」千枝子は、おばあさんを慰めるように優しく言って、臼のそばに座りました。

「そうかい。お婆さんは、精も根も尽きてのう。力が出んのじゃ。」

千枝子は、臼の取ってを握って回し始めました。迎え団子はお米の粉なので、臼は重たいのです。ゴロゴロ始めると、瑞枝がそばへやってきて、「お姉ちゃん、私もやるわ」と、すぐに手をかけました。瑞枝はまた碾き臼に慣れないけれど、それでも二人して回すと、臼は半分の重たさになります。

「勉強せえ、勉強せえ、つらいことでもがまんして―。」臼が歌い始めました。千枝子も瑞枝も、額にじっとり汗が出てきました。