\section{元旦に}
門松を立てることも、雑煮をたべることも、賀状を出すことも、実は、本当を言えば、なにを意味しているかよくは判らない。しかし、これだけは判っている、人間の一生が少々長すぎるので、神さまが、それを、三百六十五日ずつに区切ったのだ。そして、その区切り、区切りの階段で、人間がひと休みするということだ。

私は神さまが作ったその階段を、ずいぶんたくさん上がって来た。今年はその五十段目だ。昭和三十二年の明るい陽の光を浴びて、私はいまひと休みしている。はるか下の方の段で、私の四人の子供たちも、それぞれ新しい着物を着て、いまひと休みしている。