\section{銀杏が衣を脱ぐ時}

毎年、秋も深まって朝夕の冷え込みが厳しさを増す今時分になると、北の郷里の菩提寺の境内にある銀杏の巨木のことが気にかかる。

気にかかると言っても、その銀杏が老木だから、台風でも来たら大枝が折れやしないかと心配するのではない。今年の落葉はもう終わったか、どうか。まだなら、落葉するまでにあと何日ぐらい間あるだろうか。そう思って気が揉めるのである。

その銀杏は大木だから、葉を厚く繁らせていて、秋の黄葉は誠に見事である。それに、落葉の光景も思わず息を呑むほどのものであるらしい。私はまだ見たことがないから、予定が立つようなら、一度出かけてみてもいいと思っている。けれども、銀杏としても落葉の予測などつくわけがないだろう。

一枚や二枚の落葉なら話は別だが、この銀杏の葉は、短時間で一枚残らず落ちてしまうのだから。霜が降りたのではないかと思われるほど冷え込みのきつい、かんと晴れ渡った朝だと思って頂きたい。裏山から昇る朝日の光芒が、庫裏の屋根を乗り越えて境内へ降りてくる。

まず、銀杏の一番てっぺんに朝日が当たる。すると、暖められた葉が一枚、ひらと枝先を離れて、舞い落ちる。それを合図に、陽に浴びた葉が次から次へと落ち始める。ひっきりなしに落ちる。

銀杏は、暫しさわさわという落葉の音に包まれる。まるで分厚い黄金色の衣を足元へ脱ぎ落とすかのように、銀杏はみるみる裸になっていく。

銀杏に訊きたい。今年の落葉はいつごろになろうか。